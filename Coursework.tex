\documentclass{article}
\usepackage{url}
\title{BTS Coursework}
\date{March 2013}
\author{Serkan Seven \and Stian A. Johansen \and Cengiz Can \and Harith Abdullah Salim}
\begin{document}
\maketitle
\section{Introduction} % (fold)
\label{sec:Introduction}
A note for contributors of this document: source can be found revisioned here: \url{https://github.com/uansett/BTS-CW.git}
\subsection{Assumptions} % (fold)
\label{sub:Assumptions}

Company Ltd. has a range of basic handsets running their in-house set of software, and a small range of Smart phones running the free and open source Android system. Their technical expertise lies in basic handsets.

\subsubsection{Definitions} % (fold)
\label{ssub:Definitions}

\begin{description}
  \item[Smart phone] \hfill \\ 
  Smart phone is in this document defined as a handset capable of installing and running third-party applications obtained from an internet application store.
  \item[Dump Phone] \hfill \\ 
  A handset incapable of processing data over 2G or 3G networks, only providing abilities for making calls and sending text messages (SMS).
\end{description}
% subsubsection Definitions (end)
% subsection Assumptions (end)
% section Introduction (end)
\section{Issues} % (fold)
\label{sec:Issues}
Company Ltd. have experience on delivering handsets that might be challenged by the emerging technologies around tablets and 4G enabled Smart phones in the western market.
At this time, Company has introduced three lines of Smart phones (as defined by a handset capable of installing and running third party applications from an application store) and a range of standard issue handsets capable of connecting to 3G data sources and performing calls and sending texts onto 2G networks through operators. The issue is that customers might turn down handsets that are not "app capable", or turn to products that have more advanced technologies than Company's handsets can provide.

In some way or another the company need to change their presence in the market from being a stagnant oldie to become either a solid manufacturer of basic phones or take on the developments in smart phone handsets and compete in the race.
% section Issues (end)

\section{Brainstorming - possible paths} % (fold)
\label{sec:Brainstorming - possible paths}
\paragraph{Re-target the market} % (fold)
\label{par:Re-target the market}
One idea that comes to mind is that even though the western market, with a high GDP per capita sees a great demand for even faster over-the-air data transfer speeds and good looking tablet devices, the largest parts of the world (citation) are developing countries where even apps are seen as uneccesary and irrelevant. The infrastructure may not even be capable of supporting the latest communication technologies now, and may not be in even a decade.
These markets are filled with "dumb phones", and a valuable improvement for many in these makets are data-enabled phones that can process information over 2G or 3G networks.
% paragraph Re-target the market (end)

\paragraph{Join the smart phone race} % (fold)
\label{par:Join the smart phone race}
Joining in on creating competing smart phones has to be done intelligently. We may expect that researching and implementing 4G technologies in handsets will require a long period of time and may result in significant costs, entering a small subset of the smart phone market can be a good idea.

More specifically, escape the android platform currently in the line of products and use windows mobile. Benefits include a smaller set of devices to choose from by potential buyers, and so it will be easier to establish a niche in this market. Cheap Windows Phone 8 devices, for example are not very common, but with the market powers of Microsoft, the windows phone platform could potentially outgrow even the biggest names in the smart phone market at this time.

Another solution is to partner with a smaller smart phone handset manufacturer or operating systems provider. Blackberry has a well known brand for 3G capable phones, but they have not made a huge success in app-enabled devices ("smart phones"). 

% paragraph Join the smart phone race (end)

% section Brainstorming - possible paths (end)

\end{document}
